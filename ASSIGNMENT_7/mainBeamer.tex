\documentclass{beamer}
\usepackage{amssymb}
\usepackage{amsthm}
\usepackage{hyperref}
\usetheme{CambridgeUS}
\providecommand{\mbf}{\mathbf}
\providecommand{\pr}[1]{\ensuremath{\Pr\left(#1\right)}}
\providecommand{\qfunc}[1]{\ensuremath{Q\left(#1\right)}}
\providecommand{\sbrak}[1]{\ensuremath{{}\left[#1\right]}}
\providecommand{\lsbrak}[1]{\ensuremath{{}\left[#1\right.}}
\providecommand{\rsbrak}[1]{\ensuremath{{}\left.#1\right]}}
\providecommand{\brak}[1]{\ensuremath{\left(#1\right)}}
\providecommand{\lbrak}[1]{\ensuremath{\left(#1\right.}}
\providecommand{\rbrak}[1]{\ensuremath{\left.#1\right)}}
\providecommand{\cbrak}[1]{\ensuremath{\left\{#1\right\}}}
\providecommand{\lcbrak}[1]{\ensuremath{\left\{#1\right.}}
\providecommand{\rcbrak}[1]{\ensuremath{\left.#1\right\}}}
\usepackage{amssymb,amsmath}

\renewcommand\thesection{\arabic{section}}
\renewcommand\thesubsection{\thesection.\arabic{subsection}}
\renewcommand\thesubsubsection{\thesubsection.\arabic{subsubsection}}
\numberwithin{equation}{subsection}
% Title page details: 
\title{AI1110 ASSIGNMENT 7} 
\author{J. TUSHITA SHARVA - CS21BTECH11022}
\date{\today}
\logo{\large \LaTeX{}}


\begin{document}

% Title page frame
\begin{frame}
    \titlepage 
\end{frame}

% Remove logo from the next slides
\logo{}

% Outline frame
\begin{frame}{Outline}
    \tableofcontents
\end{frame}


\section{Question}
\begin{frame}{Question}
    \begin{block}{Papoulis Chapter 8, Question 6}
        Consider a random variable $x$ with density $f(x) = xe^{-x}U(x)$. Predict with $95\%$ confidence that the next value of x will be in the interval $(a, b)$. Show that the length $b-a$ of this interval is minimum if a and b are such that\\
        \begin{align}
            f(a) = f(b) && \pr{a < x < b} = 0.95
        \end{align}
        Find $a$ and $b$.
    \end{block}
\end{frame}

\section{Solution}
\begin{frame}{Solution}
     We shall show that if $f(x)$ is a density with a single maximum and $\pr{a \< x \< b} = \gamma$, the $b-a$ is minimum if $f(a) = f(b)$. The density $xe^{-x}U(x)$ is a special case. It suffices to show that $b-a$ is not minimum if $f(a) < f(b)$ or $f(a) > f(b)$.
\end{frame}


\begin{frame}{}

\begin{block}{Case \rom{1}: $f(a) > f(b)$}
        We can clearly see that $f'(a) > 0$ and $f'(b) < 0$, hence we can find two constants $\delta_1 > 0$ and $\delta_2 > 0$ such that $\pr{a+ \delta_1 < x < \delta_2} = \gamma$ and $f(a) < f(a + \delta_1) < f(b + \delta_2) < f(b)$.\\
    We know that $\delta_1 > \delta_2$. Hence, the length of the new interval $(a+\delta_1, b+\delta_2)$ is $b-a + (\delta_2-\delta_1)$ smaller than $b-a$.\\
\end{block}

\begin{block}{Case \rom{2}: $f(a) < f(b)$}
    Here, we can form the interval $\brak{a-\delta_1, b-\delta_2}$. Then the length of the interval becomes $b-a+ (\delta_1 - \delta_2)$. Since $\delta_1 < \delta_2$, the length is less than $b-a$. \\
    Hence for $f(a) < f(b)$ or $f(a) > f(b)$, the interval length is greater then $b-a$. Therefore the length of $b-a$ is minimum iff $f(a) = f(b)$.
\end{block}
\end{frame}

\begin{frame}{}
    \begin{alertblock}{Special case}
        \begin{align}
            f(x) = xe^{-x}\\
            \implies F(x) = 1-e^{-x}-xe^{-x}.\\
            \implies \pr{a < x < b} = e^{-a} + ae^{-a} - e^{-b} - be^{-b} = 0.95\\
        \end{align}
        \end{alertblock}
        \end{frame}

\begin{frame}{}
\begin{alertblock}
        Since f(a) = f(b), 
    \begin{align}
        ae^{-a} = be^{-b}\\
        \implies \pr{a < x < b} = e^{-a} - e^{-b} = 0.95\\
        \implies a = 0.04 and b = 4.75
        \implies interval = 4.71
    \end{align}
        
        If we set 
    \begin{align}
        0.025 = \pr{x \le a} = F(a) && 0.025 = \pr{x > b} = 1 - F(b)\\
        \imples 0.025 = 1 - e^{-a}-ae^{-a} && 0.025 = e^{-b} + be^{-b}\\
        \imples a = 0.242, b = 5.572\\
        \imples interval = 5.33
    \end{align}
    \end{alertblock}
    Hence proved a special case also.
\end{frame}

\begin{frame}{Source codes}
    \begin{block}{Download Latex source code of this pdf from:}
       \href{https://github.com/TushitaSharva/PRV_2022/blob/main/ASSIGNMENT_7/mainBeamer.tex}{Click Here}
    \end{block}
    
    \begin{block}{Download document of this presentation at:}
       \href{https://github.com/TushitaSharva/PRV_2022/blob/main/ASSIGNMENT_7/mainDoc.pdf}{Click Here}
    \end{block}
    
    \begin{block}{To download this presentation, visit:}
        \href{https://github.com/TushitaSharva/PRV_2022/blob/main/ASSIGNMENT_7/mainBeamer.pdf}{Click here}
    \end{block}
    
\end{frame}
\end{document}

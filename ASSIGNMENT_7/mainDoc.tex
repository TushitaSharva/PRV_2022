\let\negmedspace\undefined
\let\negthickspace\undefined
%\RequirePackage{amsmath}
\documentclass[journal,12pt,twocolumn]{IEEEtran}
%
% \usepackage{setspace}
%\usepackage{gensymb}
\usepackage[misc]{ifsym}
%\doublespacing
\usepackage{polynom}
%\singlespacing
%\usepackage{silence}
%Disable all warnings issued by latex starting with "You have..."
%\usepackage{graphicx}
\usepackage{amssymb}
%\usepackage{relsize}
\usepackage[cmex10]{amsmath}
%\usepackage{amsthm}
%\interdisplaylinepenalty=2500
%\savesymbol{iint}
%\usepackage{txfonts}
%\restoresymbol{TXF}{iint}
%\usepackage{wasysym}
\usepackage{amsthm}
%\usepackage{pifont}
%\usepackage{iithtlc}
% \usepackage{mathrsfs}
% \usepackage{txfonts}
\usepackage{stfloats}
% \usepackage{steinmetz}
\usepackage{bm}
% \usepackage{cite}
% \usepackage{cases}
% \usepackage{subfig}
%\usepackage{xtab}
\usepackage{longtable}
%\usepackage{multirow}
%\usepackage{algorithm}
%\usepackage{algpseudocode}
\usepackage{enumitem}
\usepackage{mathtools}
\usepackage{tikz}
\usepackage{tfrupee}
% \usepackage{circuitikz}
% \usepackage{verbatim}
%\usepackage{tfrupee}
\usepackage[breaklinks=true]{hyperref}
%\usepackage{stmaryrd}
%\usepackage{tkz-euclide} % loads  TikZ and tkz-base
%\usetkzobj{all}
\usepackage{listings}
    \usepackage{color}                                            %%
    \usepackage{array}                                            %%
    \usepackage{longtable}                                        %%
    \usepackage{calc}                                             %%
    \usepackage{multirow}                                         %%
    \usepackage{hhline}                                           %%
    \usepackage{ifthen}                                           %%
  %optionally (for landscape tables embedded in another document): %%
    \usepackage{lscape}     
% \usepackage{multicol}
% \usepackage{chngcntr}
%\usepackage{enumerate}
%\usepackage{tfrupee}

%\usepackage{wasysym}
%\newcounter{MYtempeqncnt}
\DeclareMathOperator*{\Res}{Res}
\DeclareMathOperator*{\equals}{=}
%\renewcommand{\baselinestretch}{2}
\renewcommand\thesection{\arabic{section}}
\renewcommand\thesubsection{\thesection.\arabic{subsection}}
\renewcommand\thesubsubsection{\thesubsection.\arabic{subsubsection}}

%\renewcommand\thesectiondis{\arabic{section}}
%\renewcommand\thesubsectiondis{\thesectiondis.\arabic{subsection}}
%\renewcommand\thesubsubsectiondis{\thesubsectiondis.\arabic{subsubsection}}

% correct bad hyphenation here
\hyphenation{op-tical net-works semi-conduc-tor}
\def\inputGnumericTable{}                                 %%

\lstset{
%language=C,
frame=single, 
breaklines=true,
columns=fullflexible
}
%\lstset{
%language=tex,
%frame=single, 
%breaklines=true
%}
\begin{document}

%


\newtheorem{theorem}{Theorem}[section]
\newtheorem{problem}{Problem}
\newtheorem{proposition}{Proposition}[section]
\newtheorem{lemma}{Lemma}[section]
\newtheorem{corollary}[theorem]{Corollary}
\newtheorem{example}{Example}[section]
\newtheorem{definition}[problem]{Definition}
%\newtheorem{thm}{Theorem}[section] 
%\newtheorem{defn}[thm]{Definition}
%\newtheorem{algorithm}{Algorithm}[section]
%\newtheorem{cor}{Corollary}
\newcommand{\BEQA}{\begin{eqnarray}}
\newcommand{\EEQA}{\end{eqnarray}}
\newcommand{\define}{\stackrel{\triangle}{=}}
\newcommand*\circled[1]{\tikz[baseline=(char.base)]{
    \node[shape=circle,draw,inner sep=2pt] (char) {#1};}}
\bibliographystyle{IEEEtran}
%\bibliographystyle{ieeetr}
\providecommand{\mbf}{\mathbf}
\providecommand{\pr}[1]{\ensuremath{\Pr\left(#1\right)}}
\providecommand{\qfunc}[1]{\ensuremath{Q\left(#1\right)}}
\providecommand{\sbrak}[1]{\ensuremath{{}\left[#1\right]}}
\providecommand{\lsbrak}[1]{\ensuremath{{}\left[#1\right.}}
\providecommand{\rsbrak}[1]{\ensuremath{{}\left.#1\right]}}
\providecommand{\brak}[1]{\ensuremath{\left(#1\right)}}
\providecommand{\lbrak}[1]{\ensuremath{\left(#1\right.}}
\providecommand{\rbrak}[1]{\ensuremath{\left.#1\right)}}
\providecommand{\cbrak}[1]{\ensuremath{\left\{#1\right\}}}
\providecommand{\lcbrak}[1]{\ensuremath{\left\{#1\right.}}
\providecommand{\rcbrak}[1]{\ensuremath{\left.#1\right\}}}
\theoremstyle{remark}
\newtheorem{rem}{Remark}
\newcommand{\sgn}{\mathop{\mathrm{sgn}}}
\providecommand{\fourier}{\overset{\mathcal{F}}{ \rightleftharpoons}}
%\providecommand{\hilbert}{\overset{\mathcal{H}}{ \rightleftharpoons}}
\providecommand{\system}{\overset{\mathcal{H}}{ \longleftrightarrow}}
	%\newcommand{\solution}[2]{\textbf{Solution:}{#1}}
\newcommand{\solution}{\noindent \textbf{Solution: }}
\newcommand{\cosec}{\,\text{cosec}\,}
\providecommand{\dec}[2]{\ensuremath{\overset{#1}{\underset{#2}{\gtrless}}}}
\newcommand{\myvec}[1]{\ensuremath{\begin{pmatrix}#1\end{pmatrix}}}
\newcommand{\mydet}[1]{\ensuremath{\begin{vmatrix}#1\end{vmatrix}}}
% %\numberwithin{equation}{section}
% \numberwithin{figure}{section}
% \numberwithin{table}{section}
\numberwithin{equation}{subsection}
% \numberwithin{problem}{section}
% \numberwithin{definition}{section}
\makeatletter
\@addtoreset{figure}{problem}
\makeatother
\let\StandardTheFigure\thefigure
\let\vec\mathbf
%\renewcommand{\thefigure}{\theproblem.\arabic{figure}}
\renewcommand{\thefigure}{\theproblem}
%\setlist[enumerate,1]{before=\renewcommand\theequation{\theenumi.\arabic{equation}}
%\counterwithin{equation}{enumi}
%\renewcommand{\theequation}{\arabic{subsection}.\arabic{equation}}
\def\putbox#1#2#3{\makebox[0in][l]{\makebox[#1][l]{}\raisebox{\baselineskip}[0in][0in]{\raisebox{#2}[0in][0in]{#3}}}}
     \def\rightbox#1{\makebox[0in][r]{#1}}
     \def\centbox#1{\makebox[0in]{#1}}
     \def\topbox#1{\raisebox{-\baselineskip}[0in][0in]{#1}}
     \def\midbox#1{\raisebox{-0.5\baselineskip}[0in][0in]{#1}}
     
    
\title{
	AI1110 ASSIGNMENT 8
}
\author{ JANGA TUSHITA SHARVA (CS21BTECH11022)% <-this % stops a space
}	

\maketitle


\begin{abstract}
This document gives solution to Chapter 8, Question 8.6 from the Papoulis and Pillai Probability, Random Variables and Stochastic Processes Text Book.
\end{abstract}

Download Latex source code of this pdf from: 
\begin{lstlisting}
    https://github.com/TushitaSharva/PRV_2022/blob/main/ASSIGNMENT_7/mainDoc.tex
\end{lstlisting}

Download Presentation of this document at:
\begin{lstlisting}
    https://github.com/TushitaSharva/PRV_2022/blob/main/ASSIGNMENT_7/mainBeamer.pdf
\end{lstlisting}

To download this document, visit: 
\begin{lstlisting}
    https://github.com/TushitaSharva/PRV_2022/blob/main/ASSIGNMENT_7/mainDoc.pdf
\end{lstlisting}

\section{Question}
Consider a random variable $x$ with density $f(x) = xe^{-x}U(x)$. Predict with $95\%$ confidence that the next value of x will be in the interval $(a, b)$. Show that the length $b-a$ of this interval is minimum if a and b are such that\\
        \begin{align}
            f(a) = f(b) && \pr{a < x < b} = 0.95
        \end{align}
        Find $a$ and $b$.

\section{Answer}
We shall show that if $f(x)$ is a density with a single maximum and $\pr{a < x < b} = \gamma$, the $b-a$ is minimum if $f(a) = f(b)$. The density $xe^{-x}U(x)$ is a special case. It suffices to show that $b-a$ is not minimum if $f(a) < f(b)$ or $f(a) > f(b)$.\\
We can clearly see that $f'(a) > 0$ and $f'(b) < 0$, hence we can find two constants $\delta_1 > 0$ and $\delta_2 > 0$ such that $\pr{a+ \delta_1 < x < \delta_2} = \gamma$ and $f(a) < f(a + \delta_1) < f(b + \delta_2) < f(b)$.\\
 
 \pagebreak
 
Case \rom{1}: $f(a) > f(b)$\\
    We know that $\delta_1 > \delta_2$. Hence, the length of the new interval $(a+\delta_1, b+\delta_2)$ is $b-a + (\delta_2-\delta_1)$ smaller than $b-a$.\\
 

 
Case \rom{2}: $f(a) < f(b)$\\
Here, we can form the interval $\brak{a-\delta_1, b-\delta_2}$. Then the length of the interval becomes $b-a+ (\delta_1 - \delta_2)$. Since $\delta_1 < \delta_2$, the length is less than $b-a$. \\
    Hence for $f(a) < f(b)$ or $f(a) > f(b)$, the interval length is greater then $b-a$. Therefore the length of $b-a$ is minimum iff $f(a) = f(b)$.
    
\section{Special Case}    
\begin{align}
            f(x) = xe^{-x}\\
            \implies F(x) = 1-e^{-x}-xe^{-x}.
            \end{align}
            
\begin{align}
        \begin{split}
            \pr{a < x < b} = e^{-a} + ae^{-a}
            \\    
                - e^{-b} - be^{-b} = 0.95
        \end{split}
\end{align}

Since f(a) = f(b), 
    \begin{align}
        ae^{-a} = be^{-b}\\
        \implies e^{-a} - e^{-b} = 0.95\\
        \implies a = 0.04\\
        \implies b = 4.75\\
        \implies interval = 4.71
    \end{align}
    If we set  $0.025 = \pr{x \le a} = F(a)$ and $0.025 = \pr{x > b} = 1 - F(b)$\\
    \begin{align}
        \implies 0.025 = 1 - e^{-a}-ae^{-a}\\
        \implies 0.025 = e^{-b} + be^{-b}\\
        \imples a = 0.242, b = 5.572\\
        \imples interval = 5.33
    \end{align}
    Hence proved a special case also.

\end{document}

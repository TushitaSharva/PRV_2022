
\documentclass[journal, 12pt, twocolumn]{IEEEtran}

\title{ASSIGNMENT 3}

\author{Janga Tushita Sharva - CS21BTECH11022}

\usepackage{graphicx}
\usepackage{amsmath}
\usepackage{amssymb}
 \usepackage{listings}
    \usepackage{color}                                            
    \usepackage{array}                                            
    \usepackage{longtable}                                        
    \usepackage{calc}                                            
    \usepackage{multirow}                                         
    \usepackage{hhline}                                          
    \usepackage{ifthen}   

\usepackage{multicolrule}
\columnseprule=0.5pt

   
%table commands   
\def\inputGnumericTable{}

\usepackage[latin1]{inputenc}                                 
\usepackage{color}                                            
\usepackage{array}                                            
\usepackage{longtable}                                        
\usepackage{calc}                                             
\usepackage{multirow}                                         
\usepackage{hhline}                                           
\usepackage{ifthen}
\usepackage{caption} 
\captionsetup[table]{skip=3pt}  

\renewcommand{\thefigure}{\arabic{table}}
\renewcommand{\thetable}{\arabic{table}}       


\providecommand{\brak}[1]{\ensuremath{\left(#1\right)}}
\renewcommand\thesection{\arabic{section}}
\renewcommand\thesubsection{\thesection.\arabic{subsection}}
\renewcommand\thesubsubsection{\thesubsection.\arabic{subsubsection}}

\newcommand{\graph}{\noindent \textbf{Graph: }}
\newcommand{\calc}{\noindent \textbf{Calculations: }}
\numberwithin{equation}{subsection}

\renewcommand{\thetable}{\theenumi}

\lstset{
frame=single, 
breaklines=true,
columns=fullflexible
}

\begin{document}

\maketitle

\begin{abstract}
This assignment contains the solution to the Example 7 from the Chapter STATISTICS, from class 9 CBSE board syllabus.
\end{abstract}


\section{Question}

A teacher wanted to analyse the performance of two sections of students
in a mathematics test of 100 marks. Looking at their performances, she found that a
few students got under 20 marks and a few got 70 marks or above. So she decided to
group them into intervals of varying sizes as follows: 0 - 20, 20 - 30, . . ., 60 - 70,
70 - 100. Then she formed the following table:

\begin{table}[ht!]
	\input{questionTable.tex}
\end{table}

A histogram for this table was prepared by a student.

\begin{figure}[h!]
	\centering
	\includegraphics[width = \columnwidth]{questionHistogram.png}
	\caption{Given histogram}
	\label{Figure_1}
	\end{figure}

Carefully examine this graphical representation. Do you think that it correctly represents
the data?

\section{Solution}

No, the given histogram is incorrect.\\
The area of the rectangles is proportional to the frequencies in histogram. In this histogram,\\

Area for class 70 - 100:
\begin{align}
    8 \times 30 = 240
\end{align}

Area for class 60 - 70:
\begin{align}
    15 \times 10 = 150
\end{align}

Thus we see that the frequency of students who obtained 60 above marks are less than the frequency of students who obtained 70 above marks, which is wrong. \\
So, make the following modifications in the lengths of the rectangles so that the areas are again proportional to the frequencies.
\begin{enumerate}
    \item Find the minimum class size. Here it is 10. 
    \item The lengths of the rectangles are then modified to be proportionate to the
class-size 10.
\end{enumerate}

Use the following concept:
\begin{align}
	\begin{split}
	    length of rectangle =
\\
        \frac{frequency of class}{class size} \times \brac{minimum class size}	\end{split}
\end{align}

Using it, lengths of rectangles will be the as follows.

\calc
\begin{align}
    [0,20] : \frac{7}{20} \times 10 = \frac{7}{2} = 3.5\\
    [20, 30] : \frac{10}{10} \times 10 = 10\\
    [30, 40] : \frac{10}{10} \times 10 = 10\\
    [40, 50] : \frac{20}{10} \times 10 = 20\\ 
    [50, 60] : \frac{20}{10} \times 10 = 20\\
    [60, 70] : \frac{15}{10} \times 10 = 15\\
    [70-100] : \frac{8}{30} \times 10 = \frac{8}{3} = 2.67
\end{align}

The information when tabulated is as shown below.


\begin{table}[ht!]
    \centering
	\input{AnswerTable.tex}
	\caption{The correct analysis of histogram}
\end{table}

So the correct graph would be as follows.\\
This graph has been plotted using \texttt{matplotlib, python}.
\begin{figure}[h!]
	\centering
	\includegraphics[width = \columnwidth]{pythonGenHistogram.png}
	\caption{Given histogram}
	\label{Figure_1}
	\end{figure}

\end{document}
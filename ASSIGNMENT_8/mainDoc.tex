\let\negmedspace\undefined
\let\negthickspace\undefined
%\RequirePackage{amsmath}
\documentclass[journal,12pt,twocolumn]{IEEEtran}
%
% \usepackage{setspace}
%\usepackage{gensymb}
\usepackage[misc]{ifsym}
%\doublespacing
\usepackage{polynom}
%\singlespacing
%\usepackage{silence}
%Disable all warnings issued by latex starting with "You have..."
%\usepackage{graphicx}
\usepackage{amssymb}
%\usepackage{relsize}
\usepackage[cmex10]{amsmath}
%\usepackage{amsthm}
%\interdisplaylinepenalty=2500
%\savesymbol{iint}
%\usepackage{txfonts}
%\restoresymbol{TXF}{iint}
%\usepackage{wasysym}
\usepackage{amsthm}
%\usepackage{pifont}
%\usepackage{iithtlc}
% \usepackage{mathrsfs}
% \usepackage{txfonts}
\usepackage{stfloats}
% \usepackage{steinmetz}
\usepackage{bm}
% \usepackage{cite}
% \usepackage{cases}
% \usepackage{subfig}
%\usepackage{xtab}
\usepackage{longtable}
%\usepackage{multirow}
%\usepackage{algorithm}
%\usepackage{algpseudocode}
\usepackage{enumitem}
\usepackage{mathtools}
\usepackage{tikz}
\usepackage{tfrupee}
% \usepackage{circuitikz}
% \usepackage{verbatim}
%\usepackage{tfrupee}
\usepackage[breaklinks=true]{hyperref}
%\usepackage{stmaryrd}
%\usepackage{tkz-euclide} % loads  TikZ and tkz-base
%\usetkzobj{all}
\usepackage{listings}
    \usepackage{color}                                            %%
    \usepackage{array}                                            %%
    \usepackage{longtable}                                        %%
    \usepackage{calc}                                             %%
    \usepackage{multirow}                                         %%
    \usepackage{hhline}                                           %%
    \usepackage{ifthen}                                           %%
  %optionally (for landscape tables embedded in another document): %%
    \usepackage{lscape}     
% \usepackage{multicol}
% \usepackage{chngcntr}
%\usepackage{enumerate}
%\usepackage{tfrupee}

%\usepackage{wasysym}
%\newcounter{MYtempeqncnt}
\DeclareMathOperator*{\Res}{Res}
\DeclareMathOperator*{\equals}{=}
%\renewcommand{\baselinestretch}{2}
\renewcommand\thesection{\arabic{section}}
\renewcommand\thesubsection{\thesection.\arabic{subsection}}
\renewcommand\thesubsubsection{\thesubsection.\arabic{subsubsection}}

%\renewcommand\thesectiondis{\arabic{section}}
%\renewcommand\thesubsectiondis{\thesectiondis.\arabic{subsection}}
%\renewcommand\thesubsubsectiondis{\thesubsectiondis.\arabic{subsubsection}}

% correct bad hyphenation here
\hyphenation{op-tical net-works semi-conduc-tor}
\def\inputGnumericTable{}                                 %%

\lstset{
%language=C,
frame=single, 
breaklines=true,
columns=fullflexible
}
%\lstset{
%language=tex,
%frame=single, 
%breaklines=true
%}
\begin{document}

%


\newtheorem{theorem}{Theorem}[section]
\newtheorem{problem}{Problem}
\newtheorem{proposition}{Proposition}[section]
\newtheorem{lemma}{Lemma}[section]
\newtheorem{corollary}[theorem]{Corollary}
\newtheorem{example}{Example}[section]
\newtheorem{definition}[problem]{Definition}
%\newtheorem{thm}{Theorem}[section] 
%\newtheorem{defn}[thm]{Definition}
%\newtheorem{algorithm}{Algorithm}[section]
%\newtheorem{cor}{Corollary}
\newcommand{\BEQA}{\begin{eqnarray}}
\newcommand{\EEQA}{\end{eqnarray}}
\newcommand{\define}{\stackrel{\triangle}{=}}
\newcommand*\circled[1]{\tikz[baseline=(char.base)]{
    \node[shape=circle,draw,inner sep=2pt] (char) {#1};}}
\bibliographystyle{IEEEtran}
%\bibliographystyle{ieeetr}
\providecommand{\mbf}{\mathbf}
\providecommand{\pr}[1]{\ensuremath{\Pr\left(#1\right)}}
\providecommand{\qfunc}[1]{\ensuremath{Q\left(#1\right)}}
\providecommand{\sbrak}[1]{\ensuremath{{}\left[#1\right]}}
\providecommand{\lsbrak}[1]{\ensuremath{{}\left[#1\right.}}
\providecommand{\rsbrak}[1]{\ensuremath{{}\left.#1\right]}}
\providecommand{\brak}[1]{\ensuremath{\left(#1\right)}}
\providecommand{\lbrak}[1]{\ensuremath{\left(#1\right.}}
\providecommand{\rbrak}[1]{\ensuremath{\left.#1\right)}}
\providecommand{\cbrak}[1]{\ensuremath{\left\{#1\right\}}}
\providecommand{\lcbrak}[1]{\ensuremath{\left\{#1\right.}}
\providecommand{\rcbrak}[1]{\ensuremath{\left.#1\right\}}}
\theoremstyle{remark}
\newtheorem{rem}{Remark}
\newcommand{\sgn}{\mathop{\mathrm{sgn}}}
\providecommand{\fourier}{\overset{\mathcal{F}}{ \rightleftharpoons}}
%\providecommand{\hilbert}{\overset{\mathcal{H}}{ \rightleftharpoons}}
\providecommand{\system}{\overset{\mathcal{H}}{ \longleftrightarrow}}
	%\newcommand{\solution}[2]{\textbf{Solution:}{#1}}
\newcommand{\solution}{\noindent \textbf{Solution: }}
\newcommand{\cosec}{\,\text{cosec}\,}
\providecommand{\dec}[2]{\ensuremath{\overset{#1}{\underset{#2}{\gtrless}}}}
\newcommand{\myvec}[1]{\ensuremath{\begin{pmatrix}#1\end{pmatrix}}}
\newcommand{\mydet}[1]{\ensuremath{\begin{vmatrix}#1\end{vmatrix}}}
% %\numberwithin{equation}{section}
% \numberwithin{figure}{section}
% \numberwithin{table}{section}
\numberwithin{equation}{subsection}
% \numberwithin{problem}{section}
% \numberwithin{definition}{section}
\makeatletter
\@addtoreset{figure}{problem}
\makeatother
\let\StandardTheFigure\thefigure
\let\vec\mathbf
%\renewcommand{\thefigure}{\theproblem.\arabic{figure}}
\renewcommand{\thefigure}{\theproblem}
%\setlist[enumerate,1]{before=\renewcommand\theequation{\theenumi.\arabic{equation}}
%\counterwithin{equation}{enumi}
%\renewcommand{\theequation}{\arabic{subsection}.\arabic{equation}}
\def\putbox#1#2#3{\makebox[0in][l]{\makebox[#1][l]{}\raisebox{\baselineskip}[0in][0in]{\raisebox{#2}[0in][0in]{#3}}}}
     \def\rightbox#1{\makebox[0in][r]{#1}}
     \def\centbox#1{\makebox[0in]{#1}}
     \def\topbox#1{\raisebox{-\baselineskip}[0in][0in]{#1}}
     \def\midbox#1{\raisebox{-0.5\baselineskip}[0in][0in]{#1}}
     
    
\title{
	AI1110 ASSIGNMENT 8
}
\author{ JANGA TUSHITA SHARVA (CS21BTECH11022)% <-this % stops a space
}	

\maketitle


\begin{abstract}
This document gives solution to Chapter 7, Question 7.20 from the Papoulis and Pillai Probability, Random Variables and Stochastic Processes Text Book.
\end{abstract}

Download Latex source code of this pdf from: 
\begin{lstlisting}
    https://github.com/TushitaSharva/PRV_2022/blob/main/ASSIGNMENT_8/mainDoc.tex
\end{lstlisting}

Download Presentation of this document at:
\begin{lstlisting}
    https://github.com/TushitaSharva/PRV_2022/blob/main/ASSIGNMENT_8/mainBeamer.pdf
\end{lstlisting}

To download this document, visit: 
\begin{lstlisting}
    https://github.com/TushitaSharva/PRV_2022/blob/main/ASSIGNMENT_8/mainDoc.pdf
\end{lstlisting}

\section{Question}
We place at random n points in the interval $(0,1)$ and we donate by x and y the distance from the origin to the first and last point respectively. Find $F(x)$, $F(y)$, $F(x,y)$.

\section{Answer}

The event {${\underset{\sim}{x}\leq x}$} occurs if there is at least one point in the interval ($0$,$x$); The event {${\underset{\sim}{y}\leq y}$} occurs if all the points are in the interval($0$, $y$).

    Let $A_X$ = {at least one point in ($0$, $x$)} = {$\underset{\sim}{x} \leq x$}\\
    Let $B_Y$ = no points in ($y$, $1$) \\ 
    = all points in ($0$,$y$) = $\underset{\sim}{y} \leq y$

Hence for $0 \leq x \leq 1$, $0 \leq y \leq 1$, we have:
\begin{align}
    F_X(x) = \pr{A_X} = 1 - \pr{\bar{A_X}} = 1-\brak{1-x}^n\\
    F_Y\brak{y} = \pr{B_Y} = y^n 
\end{align}

Further more, we have
\begin{align}
    \cbrak{\underset{\sim}{x} \leq x, \underset{\sim}{y} \leq y} = A_x \cdot B_y\\
    A_X\ B_Y + \bar{A_X}B_Y = B_Y
    \end{align}

Case \rom{1}: $x \leq y$ then,
    \begin{align}
            \bar{A_x}B_Y = \cbrak{all\ points\ in\ \brak{x, y}}\\
            \implies \pr{\bar{A_X}B_Y} = \brak{y-x}^n
    \end{align}
    
    Case \rom{2}: $x > y$ then,
    \begin{align}
        \bar{A_X}B_Y = \cbrak{\phi} 
    \end{align}
     Therefore, we have 
     F_{XY}\brak{x, y} = \pr{A_X B_Y} = 
     \begin{cases}
         y^{n}-\brak{y-x}^n & x \leq y\\
         y^n & x > y
     \end{cases}


\end{document}
